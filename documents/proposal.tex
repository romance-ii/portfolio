\documentclass[12pt]{article}

\usepackage{url}
\usepackage{listings}

\begin{document}

\title{Fetter Project Proposal}
\author{Rayiner Hashem}
\date{\today}
\maketitle

\section{Synopsis}

A program will be created to automatically generate foreign function
interface (FFI) bindings directly from C++ header files. The program
will implement this functionality using GCC-XML to parse C++ header
files, giving it the ability to support a wide range of C++ features.
The top priorities of the program will robust handling of C++ header
files, followed by ease of use, then simplicity of maintenance. It
will support multiple FFI back ends, but the first target will be UFFI
on CMUCL.

\section{Justification}

The inability to access modern C++ libraries from Lisp code is an
inconvenience for programmers. C++ is the most-used language on
SourceForge, featured in 15,607 projects. Among these are premier Open
Source platforms like Mozilla and KDE. All of this C++ code is
inaccessible from Lisp and represents not only lost opportunities for
sharing, but in the case of Mozilla and KDE, large platforms with
large user bases that cannot be targeted by Lisp applications. A
system that would allow Lisp code to access C++ libraries would thus
be quite beneficial to Lisp programmers, particularly to those
targeting Open Source platforms. It would not only ease development,
but it would widen the potential audience for Lisp applications.

Now, it is reasonable to question the need for another program to
automate the generation of FFI declarations. After all, many such
programs already exist. However, most of these existing programs do
not even attempt to handle C++ header files. Indeed, only a single
system, the Simplified Wrapper and Interface Generator (SWIG), offers
anything resembling proper support for C++ features. However, even
SWIG cannot handle C++ header files that require correct and complete
template parsing semantics. In any case, there are no SWIG back ends
available for free Common Lisp implementations, and the single Common
Lisp back end that does exist, for Allegro Common Lisp (ACL), lacks
C++ support entirely. Part of the reason why SWIG back ends for Lisp
and Lisp-like languages are so uncommon is that writing a SWIG back
end requires quite a bit of knowledge about the SWIG system itself, as
well as a deep understanding of the C++ type system. Further, SWIG
back ends must be written in C++,
  making them less palatable projects for Lisp programmers. For these
reasons, a creation of a new program seems justified.

\section{Development Process}

An incremental development model will be used to create the program.
This decision is driven by the author's preference for this model when
developing in dynamic languages and by the desire to make several
stages of intermediate results available to demonstrate progress. It
will be developed with much hindsight, as the author implemented a
proof-of-concept of a similar system in 2004. The previous system was
written in Python, targeted the Functional Developer Dylan FFI, and
was complete enough to generate declarations for C libraries like SDL
and OpenGL. Development on the concept was stopped largely because of
job-related time pressures.

\section{Project Details}

The basic structure and operation of the program can be likened to
that of a compiler. It will be run with a simple input file specifying
the C++ header files and some configuration options. A front end will
use GCC-XML to parse the header files, run several transformations on
the resulting information, then pass the results via an intermediate
file to a specified back end. The back end will then process the
output and generate a Lisp package containing the FFI declarations and
stub routines required to access the original header files.

The front end will be responsible for processing the input file,
generating an IR from the specified header files, and transforming the
IR to make it readily digestible by the back ends. Generating the IR
is fairly simple. When invoked on a header file, GCC-XML generates an
XML file containing the header file's class and field declarations.
This XML file can then be parsed into an object tree with the same
structure. Since the GCC-XML format is very close to a proper semantic
representation of the header file, this object tree can be used
directly as the IR. Having parsed the XML files and generated the IR,
the front end must then simplify the IR before it is passed to the
back ends. It must perform transformations such as collapsing
namespace hierarchies and nested structures, assigning names to
anonymous types, and filling in values for default arguments. These
transformations all exist to minimize the complexity of the back ends.
After the simplification passes, the object
 tree can be written to an output XML file and the back end can be started.

The back end will be responsible for taking the simplified IR and
generating from it FFI declarations and stub routines. Multiple back
ends can exist, each targeting a different FFI. The back ends can be
very simple recursive-descent code generators. The front end does most
of the work of lowering the IR to an appropriate level of abstraction.
This simplicity is an important trait, as it is desired that back ends
eventually be written for languages such as Dylan and Scheme. A survey
of various FFIs suggests that not all will be able to offer the same
level of access to C++ features. Three levels of support have been
identified: L0, L1, and L2. L0 access will feature standard C access
plus the ability to call methods in C++ classes. L1 access will
feature the additional ability to override virtual functions in C++
classes. L2 access will feature the ability to integrate C++ classes
into the native object system. A survey of existing FFIs suggests that
UFFI can support L0 access by itself, or L1 access with compiler-specific workarounds for
specifying callbacks. Sophisticated FFIs like the one in ACL or
Functional Developer can support L2 access.

The runtime library will be a small C support library specific to each
C++ compiler. It is necessary because existing Lisp FFIs have no
understanding of C++ virtual function tables. The runtime library will
encapsulate all the compiler-specific knowledge of virtual table
formats, allowing Lisp programs to make virtual function calls and to
generate virtual function tables that can be called from the C++ code.
    
\section{Project Time Line}

The project shall be divided into three phases, with a milestone at
the end of each phase. The phases are outlined below:

\begin{itemize}
\item Phase 1 (Mid-June to July 1): Research the target FFI and the IA64 C++
ABI (the C++ ABI for most compilers on *NIX/x86).

\item Milestone 1 (July 1): The creation of a detailed design document
specifying the structure of the program, the format of the IR, the IR
transformations in the front end, the precise semantics of the various
levels of access, and the IR to FFI mapping.

\item Phase 2 (July 1 to July 15): Create an initial implementation of the
front-end and a back end implementation capable of performing L0
access.

\item Milestone 2 (July 15): The creation of a Lisp demo using the SDL C++ API.

\item Phase 3 (July 16 to August 15): Improve the implementation of the
front end and create a back end capable of supporting L1 access.

\item Milestone 3 (August 15): The creation of a Lisp demo using the KDE C++ API.
\end{itemize}

While this plan is a bit aggressive, experience with the previous
concept suggests that it is realistic. The previous concept, as
described, was implemented in two weeks.

\section{Project Deliverables}

As a proof of completion, the following items will be delivered:

\begin{itemize}
\item A front end capable of parsing input from GCC-XML 0.60.
\item A back end capable of allowing L1 access through UFFI on CMUCL.
\item A runtime library for GCC 3.x on Linux/x86.
\end{itemize}

\end{document}


